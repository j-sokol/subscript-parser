\documentclass[11pt]{article}
\usepackage[a4paper, hmargin={2.8cm, 2.8cm}, vmargin={2.5cm, 2.5cm}]{geometry}
\usepackage{eso-pic} % \AddToShipoutPicture
\usepackage{graphicx} % \includegraphics
\usepackage{listings}




\usepackage{amsfonts}
\usepackage{amsmath}
\usepackage[pdftex]{color,graphicx}
% \usepackage{wrapfig}
% \usepackage{caption}
% \usepackage{subcaption}
% \usepackage{float}
% \usepackage{mdwlist}
% \usepackage{pgfgantt}
% \usepackage[showframe=false]{geometry}
% \usepackage{changepage}
% \usepackage[bookmarks]{hyperref}
% \usepackage{listings}
% \usepackage{dirtree}
% \usepackage{todonotes}

%% Change `ku-farve` to `nat-farve` to use SCIENCE's old colors or
%% `natbio-farve` to use SCIENCE's new colors and logo.
\def \ColourPDF {include/ku-farve}

%% Change `ku-en` to `nat-en` to use the `Faculty of Science` header
\def \TitlePDF   {include/ku-en}  % University of Copenhagen

\title{
  \vspace{3cm}
  \Huge{AP Assignment 2 - SubScript Parser} \\
  \Large{DIKU University of Copenhagen 2017}
}

\author{
  \Large{Jan Sokol, Denis Trebula}
  \\ \texttt{\{kgj360,jmp640\}@alumni.ku.dk} \\
}

\date{
    \today
}

\begin{document}


\AddToShipoutPicture*{\put(0,0){\includegraphics*[viewport=0 0 700 600]{\ColourPDF}}}
\AddToShipoutPicture*{\put(0,602){\includegraphics*[viewport=0 600 700 1600]{\ColourPDF}}}

\AddToShipoutPicture*{\put(0,0){\includegraphics*{\TitlePDF}}}

\clearpage\maketitle
\thispagestyle{empty}

\newpage



%% Write your dissertation here.

\section{Subscript Parser} \\


\section{Choice of parser combinatory library} \\

At first we had to choose a solid library for working with parsing and grammar in our project. Two main
libraries which got our attention were Parsec and ReadP. After further decision making we chose to use
the Parsec combinatory library because according to the documentation it performs best on predictive
(LL) grammars. In addition, it contains extensive error messages which make the development of the
code easier. Finally, we can combine small parsing functions to build more complex parsers.
\section{Usage of biased combinators} \\

When we need to parse a text for example basic true or false text, we try applying it to a parser
function. If we didn’t use the try combinator, we could lose the already parsed text and none of the
other parsers would have given a correct result.
 With the try combinator, we have the ability to backtrack and pass the same text to another
parser. This is needed when we parse the SubScript language because many of its grammar productions
have many alternatives. If a parser fails for an one alternative, then we have the ability to try the same
text for the next alternative.
\section{Grammar} \\

To have grammar that is working with Parsec library, we have to remove left recursion. If a context-free grammar
has a $A  \rightarrow A\alpha$ rule, then is left-recursive. In grammar given in assesement, we can see this in first rule for example:
\begin{verbatim}
Expr ::= Expr ‘,’ Expr | Expr1
\end{verbatim}

To remove left recursion from grammar, we need to replace left-recursive rules.[1][2] Rule $A  \rightarrow A\alpha|\beta$ can be rewritten as:
\\
$A \rightarrow \beta A'$ \\
$A' \rightarrow aA'$

Also as stated in the assesement, some operators precedence the others. For example \texttt{*, \%} have higher precedence than \texttt{+, -}, so we split \texttt{Expr1} rule into multiple rules.

Using this we transformed grammar from assessement to:
\begin{verbatim}
  Expr        ::= Expr1 Expr'

  Expr'       ::= ',' Expr | eps

  Expr1       ::= Ident '=' Expr1
              | Expr2

  Expr2       ::= Expr3 Expr2'

  Expr2'      ::= '===' Expr3 Expr2'
              | '<' Expr3 Expr2'
              | eps

  Expr3       ::= Expr4 Expr3'

  Expr3'    ::= '+' Expr4 Expr3'
              | '-' Expr4 Expr3'
              | eps

  Expr4       ::= ExprTerm Expr4'

  Expr4'    ::= '*' ExprTerm Expr4'
              | '%' ExprTerm Expr4'
              | eps

  ExprTerm  ::= Number
              | String
            	| 'true'
            	| 'false'
            	| 'undefined'
            	| Ident ExprIdent
            	| '[' Exprs ']'
            	| '[' 'for' '(' Ident 'of' Expr1 ')' ArrayCompr Expr1 ']'

  ExprIdent ::= '(' Exprs ')'
              | eps

  Exprs       ::= Expr1 CommaExprs
              | eps

  CommaExprs  ::= ',' Expr1 CommaExprs
              | eps

  ArrayFor    ::= 'for' '(' Ident 'of' Expr1 ')' ArrayCompr

  ArrayIf     ::= 'if' '(' Expr1 ')'

  ArrayCompr  ::= Expr1
              | ArrayFor
              | ArrayIf

  Ident       ::= (as in assignment)
  Number      ::=
  String      ::=
\end{verbatim}

\section{Whitespace handling} \\

For optimization purposes we handle whitespaces using three functions. To remove any unnecesary whitespaces around string
 token we use function \texttt{sToken} that takes string as a parameter. (Function uses \texttt{spaces} parser from Parsec library and removes any amount of whitespaces, that would make parser fail otherwise.)
\begin{verbatim}
  sToken :: String -> Parser String
  sToken s = try (spaces >> string s <* spaces)
\end{verbatim}
Second function \texttt{whitespaceParser} is more robust, also calls funtion \texttt{commentHandler} if necesary. Is used mainly in function
\texttt{backslashCheck} that parses strings. For easier measures we also used function whitespaceHandler which can be found in slides from second lecture on parser.
\begin{verbatim}

whitespaceParser :: Parser ()
whitespaceParser = do
                    first <- many $ oneOf " \n\t"
                    second <- many commentHandler
                    when (not (null first) || not (null second)) $
                      do whitespaceParser -- recursive call
                         return ()
\end{verbatim}





\section{Parser implementation and Assesment} \\

For implementation we used skeleton from the assesment zip file. We changed type of ParserError from one defined in assesment to one in \texttt{Text.Parsec.Error} library. 

Based on our tests and OnlineTA we can say that our parser is able to succesfully parse SubScript files given in the assesment directory. 
Parser only has problems with comments, if they are inside string (such as name of a variable).

 Precedence of operators is correctly implemented (acording to our tests and tests from OnlineTA), also operators have left association
 (we have included tests for this). If a number has more than 8 digits (or 7 with negative sign), parser won't accept it.

\section{How to run code and reproduce test results} \\

We have managed to implement a bit of unit testing in the \texttt{Parser/Tests.hs}. We used tasty framework
which was widely recommended by the Haskell community. Take in mind that in order to properly compile
and run tests, you must have downloaded and installed tasty framework and its plugin,
\texttt{tasty-HUnit}. In order to do so type this in command line:
\begin{verbatim}

 stack install tasty
 stack install tasty-hunit
 \end{verbatim}

To run the code with our unit tests:
\begin{verbatim}

 stack exec ghci -- -W Parser/Tests.hs
 \end{verbatim}

And run the main function as:
\begin{verbatim}

 main
 \end{verbatim}

This function will perform all the tests in the Tests.hs module. The tests evaluate a given input to a
parser if it is the same as the expected output. If the result matches the expected value, then the tests
are successful.

\subsection{Resources}
\texttt{[1]} \href{http://www.csd.uwo.ca/~moreno//CS447/Lectures/Syntax.html/node8.html}
\\
\texttt{[2]} \href{http://web.eecs.umich.edu/~weimerw/2009-4610/reading/left-recursion.pdf}
\end{document}
